%\chapter*{}
%\thispagestyle{empty}
%\cleardoublepage

%\thispagestyle{empty}

%\begin{titlepage}
 
\thispagestyle{empty}
\setlength{\centeroffset}{-0.5\oddsidemargin}
\addtolength{\centeroffset}{0.5\evensidemargin}
\thispagestyle{empty}

\noindent\hspace*{\centeroffset}\begin{minipage}{\textwidth}

\centering
%\includegraphics[width=0.9\textwidth]{images/logo_ugr.png}\\[1.4cm]



 \vspace{3.3cm}

%si el proyecto tiene logo poner aquí
\includegraphics{images/mm.pdf} 
 \vspace{0.5cm}

% Title

{\Huge\bfseries Geometría y visualización\\
}
%\noindent\rule[-1ex]{\textwidth}{3pt}\\[3.5ex]
%{\large\bfseries Subtítulo del proyecto.\\[4cm]}
\end{minipage}

\vspace{2.5cm}
\noindent\hspace*{\centeroffset}\begin{minipage}{\textwidth}
\centering

\textbf{Autor}\\ {Jesús Bueno Urbano}\\[2.5ex]
\textbf{Directores}\\
{Carlos Ureña Almagro\\
Pedro A. García Sánchez}\\[2cm]
%\includegraphics[width=0.15\textwidth]{images/tstc.png}\\[0.1cm]
%\textsc{Departamento de Teoría de la Señal, Telemática y Comunicaciones}\\
%\textsc{---}\\
%Granada, mes de 201
\end{minipage}
%\addtolength{\textwidth}{\centeroffset}
\vspace{\stretch{2}}

 
\end{titlepage}






\cleardoublepage
\thispagestyle{empty}

\begin{center}
{\large\bfseries Geometría y visualización}\\
\end{center}
\begin{center}
Jesús Bueno Urbano\\
\end{center}

%\vspace{0.7cm}
%\noindent{\textbf{Palabras clave}: palabra\_clave1, palabra\_clave2, palabra\_clave3, ......}\\

\vspace{0.7cm}
\noindent{\textbf{Resumen}}

En este Trabajo Fin de Máster hemos querido hacer un recorrido por los conceptos básicos de del Ray Tracing para poder representar y visualizar superfícies implícitas. Para ello en el  primer capítulo haremos un repaso sobre métodos elementales para implicitar superficies, es decir, para poder transformar una expresión no implícita, paramétrica en nuestro caso, de una superficie en la expresión implícita de ésta. Sobre todo nos centraremos en el método de la base de Gröbner y presentaremos los métodos de la resultante y de Wu-Ritt.

Ya en el segundo capítulo el tema principal de éste será buscar métodos de visualización de superficies como la poligonalización de superficies, en particular, presentaremos un algoritmo de triangulación extraído de \cite{Hartmann03}. A continuación explicaremos el concepto de Ray Tracing.

A continuación haremos una introducción al Análisis de Intervalos y a sus propiedades, a los Intervalos Modales y a las llamadas extensiones semánticas. Todo esto nos servirá para, finalmente, aplicarlo al Ray Tracing y así presentar varios algoritmos mejorados cuya eficiencia compararemos.
\cleardoublepage


%\thispagestyle{empty}


%\begin{center}
%{\large\bfseries Project Title: Project Subtitle}\\
%\end{center}
%\begin{center}
%First name, Family name (student)\\
%\end{center}

%\vspace{0.7cm}
%\noindent{\textbf{Keywords}: Keyword1, Keyword2, Keyword3, ....}\\

%\vspace{0.7cm}
%\noindent{\textbf{Abstract}}\\

%Write here the abstract in English.

\chapter*{}
\thispagestyle{empty}

\noindent{\rule[-1ex]{\textwidth}{2pt}\\[4.5ex]}
Yo, \textbf{Jesús Bueno Urbano}, estudiante del Máster Universitario en Matemáticas de la \textbf{Escuela Internacional de Posgrado de la Universidad de Granada}, con DNI 20078941X, autorizo que la siguiente copia de mi Trabajo Fin de Máster pueda ser consultada por las personas que lo deseen.

\vspace{2cm}

\includegraphics[scale=0.3]{images/firma.png}

\vspace{1cm}

\noindent Fdo: Jesús Bueno Urbano

\vspace{2cm}

\begin{flushright}
En Granada, a \today.
\end{flushright}


\chapter*{Agradecimientos}
\thispagestyle{empty}
\vspace{1cm}

Agradecimientos aquí.